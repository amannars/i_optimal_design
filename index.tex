% Options for packages loaded elsewhere
\PassOptionsToPackage{unicode}{hyperref}
\PassOptionsToPackage{hyphens}{url}
\PassOptionsToPackage{dvipsnames,svgnames,x11names}{xcolor}
%
\documentclass[
  letterpaper,
  DIV=11,
  numbers=noendperiod]{scrreprt}

\usepackage{amsmath,amssymb}
\usepackage{iftex}
\ifPDFTeX
  \usepackage[T1]{fontenc}
  \usepackage[utf8]{inputenc}
  \usepackage{textcomp} % provide euro and other symbols
\else % if luatex or xetex
  \usepackage{unicode-math}
  \defaultfontfeatures{Scale=MatchLowercase}
  \defaultfontfeatures[\rmfamily]{Ligatures=TeX,Scale=1}
\fi
\usepackage{lmodern}
\ifPDFTeX\else  
    % xetex/luatex font selection
\fi
% Use upquote if available, for straight quotes in verbatim environments
\IfFileExists{upquote.sty}{\usepackage{upquote}}{}
\IfFileExists{microtype.sty}{% use microtype if available
  \usepackage[]{microtype}
  \UseMicrotypeSet[protrusion]{basicmath} % disable protrusion for tt fonts
}{}
\makeatletter
\@ifundefined{KOMAClassName}{% if non-KOMA class
  \IfFileExists{parskip.sty}{%
    \usepackage{parskip}
  }{% else
    \setlength{\parindent}{0pt}
    \setlength{\parskip}{6pt plus 2pt minus 1pt}}
}{% if KOMA class
  \KOMAoptions{parskip=half}}
\makeatother
\usepackage{xcolor}
\setlength{\emergencystretch}{3em} % prevent overfull lines
\setcounter{secnumdepth}{5}
% Make \paragraph and \subparagraph free-standing
\makeatletter
\ifx\paragraph\undefined\else
  \let\oldparagraph\paragraph
  \renewcommand{\paragraph}{
    \@ifstar
      \xxxParagraphStar
      \xxxParagraphNoStar
  }
  \newcommand{\xxxParagraphStar}[1]{\oldparagraph*{#1}\mbox{}}
  \newcommand{\xxxParagraphNoStar}[1]{\oldparagraph{#1}\mbox{}}
\fi
\ifx\subparagraph\undefined\else
  \let\oldsubparagraph\subparagraph
  \renewcommand{\subparagraph}{
    \@ifstar
      \xxxSubParagraphStar
      \xxxSubParagraphNoStar
  }
  \newcommand{\xxxSubParagraphStar}[1]{\oldsubparagraph*{#1}\mbox{}}
  \newcommand{\xxxSubParagraphNoStar}[1]{\oldsubparagraph{#1}\mbox{}}
\fi
\makeatother


\providecommand{\tightlist}{%
  \setlength{\itemsep}{0pt}\setlength{\parskip}{0pt}}\usepackage{longtable,booktabs,array}
\usepackage{calc} % for calculating minipage widths
% Correct order of tables after \paragraph or \subparagraph
\usepackage{etoolbox}
\makeatletter
\patchcmd\longtable{\par}{\if@noskipsec\mbox{}\fi\par}{}{}
\makeatother
% Allow footnotes in longtable head/foot
\IfFileExists{footnotehyper.sty}{\usepackage{footnotehyper}}{\usepackage{footnote}}
\makesavenoteenv{longtable}
\usepackage{graphicx}
\makeatletter
\def\maxwidth{\ifdim\Gin@nat@width>\linewidth\linewidth\else\Gin@nat@width\fi}
\def\maxheight{\ifdim\Gin@nat@height>\textheight\textheight\else\Gin@nat@height\fi}
\makeatother
% Scale images if necessary, so that they will not overflow the page
% margins by default, and it is still possible to overwrite the defaults
% using explicit options in \includegraphics[width, height, ...]{}
\setkeys{Gin}{width=\maxwidth,height=\maxheight,keepaspectratio}
% Set default figure placement to htbp
\makeatletter
\def\fps@figure{htbp}
\makeatother
% definitions for citeproc citations
\NewDocumentCommand\citeproctext{}{}
\NewDocumentCommand\citeproc{mm}{%
  \begingroup\def\citeproctext{#2}\cite{#1}\endgroup}
\makeatletter
 % allow citations to break across lines
 \let\@cite@ofmt\@firstofone
 % avoid brackets around text for \cite:
 \def\@biblabel#1{}
 \def\@cite#1#2{{#1\if@tempswa , #2\fi}}
\makeatother
\newlength{\cslhangindent}
\setlength{\cslhangindent}{1.5em}
\newlength{\csllabelwidth}
\setlength{\csllabelwidth}{3em}
\newenvironment{CSLReferences}[2] % #1 hanging-indent, #2 entry-spacing
 {\begin{list}{}{%
  \setlength{\itemindent}{0pt}
  \setlength{\leftmargin}{0pt}
  \setlength{\parsep}{0pt}
  % turn on hanging indent if param 1 is 1
  \ifodd #1
   \setlength{\leftmargin}{\cslhangindent}
   \setlength{\itemindent}{-1\cslhangindent}
  \fi
  % set entry spacing
  \setlength{\itemsep}{#2\baselineskip}}}
 {\end{list}}
\usepackage{calc}
\newcommand{\CSLBlock}[1]{\hfill\break\parbox[t]{\linewidth}{\strut\ignorespaces#1\strut}}
\newcommand{\CSLLeftMargin}[1]{\parbox[t]{\csllabelwidth}{\strut#1\strut}}
\newcommand{\CSLRightInline}[1]{\parbox[t]{\linewidth - \csllabelwidth}{\strut#1\strut}}
\newcommand{\CSLIndent}[1]{\hspace{\cslhangindent}#1}

\KOMAoption{captions}{tableheading}
\makeatletter
\@ifpackageloaded{bookmark}{}{\usepackage{bookmark}}
\makeatother
\makeatletter
\@ifpackageloaded{caption}{}{\usepackage{caption}}
\AtBeginDocument{%
\ifdefined\contentsname
  \renewcommand*\contentsname{Table of contents}
\else
  \newcommand\contentsname{Table of contents}
\fi
\ifdefined\listfigurename
  \renewcommand*\listfigurename{List of Figures}
\else
  \newcommand\listfigurename{List of Figures}
\fi
\ifdefined\listtablename
  \renewcommand*\listtablename{List of Tables}
\else
  \newcommand\listtablename{List of Tables}
\fi
\ifdefined\figurename
  \renewcommand*\figurename{Figure}
\else
  \newcommand\figurename{Figure}
\fi
\ifdefined\tablename
  \renewcommand*\tablename{Table}
\else
  \newcommand\tablename{Table}
\fi
}
\@ifpackageloaded{float}{}{\usepackage{float}}
\floatstyle{ruled}
\@ifundefined{c@chapter}{\newfloat{codelisting}{h}{lop}}{\newfloat{codelisting}{h}{lop}[chapter]}
\floatname{codelisting}{Listing}
\newcommand*\listoflistings{\listof{codelisting}{List of Listings}}
\makeatother
\makeatletter
\makeatother
\makeatletter
\@ifpackageloaded{caption}{}{\usepackage{caption}}
\@ifpackageloaded{subcaption}{}{\usepackage{subcaption}}
\makeatother

\ifLuaTeX
  \usepackage{selnolig}  % disable illegal ligatures
\fi
\usepackage{bookmark}

\IfFileExists{xurl.sty}{\usepackage{xurl}}{} % add URL line breaks if available
\urlstyle{same} % disable monospaced font for URLs
\hypersetup{
  pdftitle={i\_optimal\_book},
  pdfauthor={Stu and Company},
  colorlinks=true,
  linkcolor={blue},
  filecolor={Maroon},
  citecolor={Blue},
  urlcolor={Blue},
  pdfcreator={LaTeX via pandoc}}


\title{i\_optimal\_book}
\author{Stu and Company}
\date{2025-03-02}

\begin{document}
\maketitle

\renewcommand*\contentsname{Table of contents}
{
\hypersetup{linkcolor=}
\setcounter{tocdepth}{2}
\tableofcontents
}

\bookmarksetup{startatroot}

\chapter*{Preface}\label{preface}
\addcontentsline{toc}{chapter}{Preface}

\markboth{Preface}{Preface}

\textbf{I-optimal Design of Experiments Using R}

\begin{enumerate}
\def\labelenumi{\arabic{enumi}.}
\item
  Preface
\item
  A Short Introduction to R

  \begin{enumerate}
  \def\labelenumii{\arabic{enumii}.}
  \item
    R, R Studio, and user-written libraries
  \item
    Data types
  \item
    Reading and writing data
  \item
    Operations with vectors and matrices
  \item
    Logical operators
  \item
    Base R graphics
  \item
    Selected R libraries (plot3D, mix.DOE, and the optimization one)
  \end{enumerate}
\item
  Response Surface Models

  \begin{enumerate}
  \def\labelenumii{\arabic{enumii}.}
  \item
    Amount models (Cartesian space)
  \item
    Mixture models (Simplex space)
  \item
    Mixture-amount models (Combined space)
  \item
    Mixture-process models (Combined space)
  \item
    Mixture-amount-process models (Combined space)
  \end{enumerate}
\item
  Constructing I-optimal Designs

  \begin{enumerate}
  \def\labelenumii{\arabic{enumii}.}
  \item
    Point optimization

    \begin{enumerate}
    \def\labelenumiii{\arabic{enumiii}.}
    \item
      Weighting matrix construction
    \item
      Space matrix construction
    \item
      Point selection and optimization
    \end{enumerate}
  \item
    Pick-and-exchange algorithms
  \item
    Design visualization using R
  \end{enumerate}
\item
  Experiments in 2-D Cartesian Space

  \begin{enumerate}
  \def\labelenumii{\arabic{enumii}.}
  \item
    Unconstrained spaces
  \item
    Constrained spaces
  \end{enumerate}
\item
  Experiments in 3-D Cartesian Space

  \begin{enumerate}
  \def\labelenumii{\arabic{enumii}.}
  \item
    Unconstrained spaces
  \item
    Constrained spaces
  \end{enumerate}
\item
  Experiments in 4-D and Higher Cartesian Spaces

  \begin{enumerate}
  \def\labelenumii{\arabic{enumii}.}
  \item
    Unconstrained spaces
  \item
    Constrained spaces
  \end{enumerate}
\item
  3-Component Mixtures

  \begin{enumerate}
  \def\labelenumii{\arabic{enumii}.}
  \item
    Unconstrained spaces
  \item
    Constrained spaces
  \end{enumerate}
\item
  4-Component Mixtures

  \begin{enumerate}
  \def\labelenumii{\arabic{enumii}.}
  \item
    Unconstrained spaces
  \item
    Constrained spaces
  \end{enumerate}
\item
  5-Component and Higher Mixtures

  \begin{enumerate}
  \def\labelenumii{\arabic{enumii}.}
  \item
    Unconstrained spaces
  \item
    Constrained spaces
  \end{enumerate}
\item
  Mixture Experiments in the Complete Simplex

  \begin{enumerate}
  \def\labelenumii{\arabic{enumii}.}
  \item
    3-component mixtures
  \item
    4-component mixtures
  \item
    5- and higher component mixtures
  \end{enumerate}
\item
  Mixture Experiments in the Constrained Simplex

  \begin{enumerate}
  \def\labelenumii{\arabic{enumii}.}
  \item
    3-component mixtures
  \item
    4-component mixtures
  \item
    5- and higher component mixtures~
  \item
    Design construction using R~

    \begin{enumerate}
    \def\labelenumiii{\arabic{enumiii}.}
    \item
      Point optimization
    \item
      Pick-and-exchange
    \end{enumerate}
  \item
    Design visualization using R~
  \end{enumerate}
\item
  Mixture-Amount Experiments

  \begin{enumerate}
  \def\labelenumii{\arabic{enumii}.}
  \item
    3-component mixture-amount experiments
  \item
    4-component mixture-amount experiments
  \item
    5- and higher component mixture-amount experiments~
  \item
    Design construction using R~

    \begin{enumerate}
    \def\labelenumiii{\arabic{enumiii}.}
    \item
      Point optimization
    \item
      Pick-and-exchange
    \end{enumerate}
  \item
    Design visualization using R
  \end{enumerate}
\item
  Constrained Mixture-Amount Experiments
\item
  Mixture-Processing Experiments
\item
  Mixture-Amount-Processing Experiments
\item
  Designs using Non-linear Response Surface Models
\end{enumerate}

Appendix 1 -- Design Visualization Using ggplot

\bookmarksetup{startatroot}

\chapter{A Short Introduction to R}\label{a-short-introduction-to-r}

R is a powerful language and environment for statistical computing and
graphics. It is a GNU project which is similar to the S language and
environment which was developed at Bell Laboratories (formerly AT\&T,
now Lucent Technologies) by John Chambers and colleagues. R can be
considered as a different implementation of S. There are some important
differences, but much code written for S runs unaltered under R.

\bookmarksetup{startatroot}

\chapter{Response Surface Models}\label{response-surface-models}

Response Surface Models (RSM) are used to predict the response of a
system to different input variables. The RSM is a mathematical model
that describes the relationship between the response and the input
variables. The RSM is used to optimize the system by finding the input
variables that will produce the desired response. The RSM is a useful
tool for engineers and scientists who need to optimize a system with
multiple input variables.

\bookmarksetup{startatroot}

\chapter*{References}\label{references}
\addcontentsline{toc}{chapter}{References}

\markboth{References}{References}

\phantomsection\label{refs}
\begin{CSLReferences}{0}{1}
\end{CSLReferences}




\end{document}
